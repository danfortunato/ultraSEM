\documentclass[11pt,a4paper,review]{siamart171218}
\usepackage[utf8]{inputenc}
\usepackage[english]{babel}

\usepackage{tikz}
\usetikzlibrary{shapes.geometric}

\usepackage{graphicx}
\usepackage{fancyhdr}
\usepackage{dsfont}
\usepackage{stmaryrd}

\usepackage{amsfonts}
\usepackage{amssymb}
%\usepackage{amsthm}

\usepackage{amsmath}
\DeclareMathOperator*{\argmax}{arg\,max}
\DeclareMathOperator*{\argmin}{arg\,min}
\usepackage{amsfonts}
\usepackage{amssymb}
\usepackage{mathtools}
\usepackage{enumitem}

\usepackage{mathrsfs}

%\usepackage{algorithm2e}
%\usepackage{algorithmic}

\usepackage{algorithm}
\usepackage{algorithmic}
\usepackage{overpic} 

\usepackage{mathtools,booktabs}
\DeclarePairedDelimiter\ceil{\lceil}{\rceil}
\DeclarePairedDelimiter\floor{\lfloor}{\rfloor}

\usepackage{varwidth}

\usepackage[title]{appendix}

\newcommand{\ttrank}{{\rm rank}^{\rm TT}}
\newcommand{\mlrank}{{\rm rank}^{\rm ML}}
\newcommand{\cprank}{{\rm rank}^{\rm CP}}
\newcommand{\rank}{{\rm rank}}
\newcommand{\lex}{<_{\rm lex}}
\newcommand{\lexeq}{\le_{\rm lex}}
%\theoremstyle{definition}
%\newtheorem{definition}{Definition}[section]

%\newtheorem{theorem}{Theorem}[section]
%\newtheorem{corollary}{Corollary}[theorem]
%\newtheorem{lemma}[theorem]{Lemma}
\newtheorem{example}{Example}[section]

\usepackage{cite}

\title{The ultraspherical spectral element method\thanks{Submitted to the editors \today.
\funding{This work is supported by National Science Foundation grant no.~1818757 and the National Defense Science and Engineering Graduate Fellowship.}}}
\author{Dan Fortunato\thanks{School of Engineering and Applied Sciences, Harvard University, Cambridge, MA 02138. (\email{dfortunato@g.harvard.edu})} \and Nick Hale\thanks{} \and Alex Townsend\thanks{Department of Mathematics, Cornell University, Ithaca, NY 14853. (\email{townsend@cornell.edu})}}
\headers{ultraSEM}{Dan Fortunato, Nick Hale, and Alex Townsend}
\begin{document}
\newcommand{\R}[0]{\mathbb{R}}
\newcommand{\C}[0]{\mathbb{C}}
\maketitle

\begin{abstract}
\end{abstract}

\begin{keywords}
\end{keywords}

\begin{AMS}
\end{AMS}

\section{Introduction}\label{sec:introduction}

\subsection{Advantages}
\begin{enumerate}
\item Complexity is $\mathcal{O}(p^4/h^3)$ instead of $\mathcal{O}(p^6/h^3)$.
\item Closer to ``true" $hp$-adaptivity, since this allows for $p$ in the hundreds.
\item Stability?
\item Robust to skinny elements.
\item Storage of solution operator allows for fast solves with multiple RHSs/BCs (e.g. time-stepping).
\item Hierarchical Poincar\'{e}--Steklov can be used as an element method (i.e. it doesn't have to look like nested dissection).
\item No mapping between grids to deal with corners.
\end{enumerate}

\section*{Acknowledgements}

\bibliography{references}
\bibliographystyle{siam}

\end{document}